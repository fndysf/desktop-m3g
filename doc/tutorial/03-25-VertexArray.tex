\subsection {バーッテックスアレイ}

VertexArrayクラスはデータを保持するためのクラスです。
データは単なる数値の集合として扱われます。

1頂点は2,3,4コンポーネントからなります。
1コンポーネントは1バイト(char)か2バイト(short)です。
4バイト型(float?)はありません。
バーテックスアレイのデータはそのまま参照されるのではなく、
最終的なデータはbiasとoffsetで演算したものが最終データとなります。
biasとoffsetはVertexArrayクラスの範囲外です。

\subsubsection{コンストラクタ}

VertexArrayクラスは頂点数とコンポーネント数、コンポーネントサイズを指定してインスタンス化します。

\begin{verbatim}
    VertexArray (int num_vertices, int num_components, int component_size);
\end{verbatim}

第1引数のnum\_verticesには頂点数を、第2引数のnum\_componentsにはコンポーネント数を、
第3引数のcomponent\_sizeにはコンポーネントサイズを指定します。
頂点数は1〜65535まで対応しています。コンポーネント数は2,3,4に対応しています。
コンポーネントサイズは1か2です。
ライブラリ内部で確保された領域の値は未定義です。

現在の頂点数、コンポーネント数、コンポーネントタイプ(サイズ)は以下のメソッドで取得できます。

\begin{verbatim}
    int getVertexCount () const;
    int getComponentType () const;
    int getComponentCount () const;
\end{verbatim}

\subsubsection{値の設定}

作成したVertexArrayオブジェクトにはset()メソッドを使って値を設定します。

\begin{verbatim}
    void set (int first_vertex, int num_vertices, char* values);
    void set (int first_vertex, int num_vertices, short* values);
\end{verbatim}

第1引数にはデータの開始位置を頂点番号で指定します。
第2引数にはデータの個数を頂点数で指定します。
第3引数にはセットしたいデータのポインターを指定します。
コンポーネント数に応じてchar型とshort型があります。どちらもsigned型です。
値はライブラリ内部に確保したメモリ領域にコピーされます。

現在設定されているデータはget()メソッドを使います。

\begin{verbatim}
    void get (int first_vertex, int num_vertices, char* values) const;
    void get (int first_vertex, int num_vertices, short* values) const;
\end{verbatim}

第1引数にはデータの開始位置を頂点番号で指定します。
第2引数にはデータの個数を頂点数で指定します。
第3引数には結果を書き込むメモリ領域へのポインターを渡します。
コンポーネント数に応じてchar型とshort型があります。どちらもsigned型です。



\begin{verbatim}
    void get (int first_vertex, int num_vertices, float scale, float* bias, float* values) const;
\end{verbatim}


