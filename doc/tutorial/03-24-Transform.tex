
\subsection{トランスフォーム}

Transformはトランスフォームを表す4x4行列を抽象化したクラスです。
デフォルトは単位行列に設定されています。
Transformオブジェクトは原則(ポインターではなく)参照で渡され、
ライブラリ内部でコピーされます。

Transformを表す4x4行列を取得するにはget()メソッドを使います。

\begin{verbatim}
    void get (float* matrix) const;
\end{verbatim}

引数のmatrixにはfloat16個分の領域へのポインターを渡します。
行列は行優先(row-major)で格納されます。

\begin{pmatrix}
  0  & 1  & 2  & 3  \\
  4  & 5  & 6  & 7  \\
  8  & 9  & 10 & 11 \\
  12 & 13 & 14 & 15
\end{matrix}

明示的に4x4行列を設定するにはset()メソッドを使います。

\begin{verbatim}
    void set (const float* matrix);
    void set (const Transform& transform);
\end{verbatim}

引数には16個のfloat(row-major)で行列を指定するか、
Transformオブジェクトを指定します。

setIdentity()メソッドはこの4x4行列(M)を単位行列化します。

\[ M = I \]

\begin{verbatim}
    void setIdentity ();
\end{verbatim}

invert()メソッドはこの4x4行列を逆行列化します。
逆行列が計算できない場合は例外を発生します。

\[ M' = M^-1 \]

\begin{verbatim}
    void invert ();
\end{verbatim}


transopose()メソッドはこの4x4行列を転置します。

\[ M' = M^t \]


\begin{verbatim}
    void transpose ();
\end{verbatim}

postTranslate()メソッドは引数で指定された平行移動を表す4x4行列(T)を作成し、
この4x4行列(M)に右から乗算します。

\[ M' =  M T \]

\begin{verbatim}
    void postTranslate (float tx, float ty, float tz);
\end{verbatim}

引数には移動量(tx,ty,tz)を指定します。

postRotate()メソッドは引数で指定された回転を表す4x4行列(R)を作成し、
この4x4行列(M)に右から乗算します。

\begin{verbatim}
    void postRotate (float angle, float ax, float ay, float az);
\end{verbatim}

引数のangleには回転角度をdegreeで、ax,ay,azは回転軸(ax,zy,az)を指定します。
回転軸のベクトルは単位ベクトルである必要はありません。

postRotate()メソッドはクォータニオンで指定された回転を表す4x4行列(R)を作成し、
この4x4行列(M)に右から乗算します。

\[ M' =  M R \]

\begin{verbatim}
    void postRotateQuat (float qx, float qy, float qz, float qw);
\end{verbatim}

(qx,qy,qz)がクォータニオンの虚数成分で、qwが実数成分です。

postScale()メソッドは引数で指定された拡大縮小を表す4x4行列(S)を作成し、
この4x4行列(M)に右から乗算します。

\[ M' =  M S \]

\begin{verbatim}
    void postScale (float sx, float sy, float sz);
\end{verbatim}

引数にはX,Y,Z軸方向の拡大縮小率を指定します。1.0が現状維持です。

postMultiply()メソッドは引数で指定された4x4行列(T)を、
この4x4行列(M)に右から乗算します。

\[ M' =  M T \]

\begin{verbatim}
    void postMultiply (const Transform& transform);
\end{verbatim}


transform()メソッドはfloat4個で表されたベクトル(v)を、
この4x4行列(M)を使って変換します。

\[ v' = M v \]

\begin{verbatim}
    void transform (float* vectors) const;
\end{verbatim}

引数のvectorsは変換結果で上書きされます。


またtransform()メソッドはVertexArrayオブジェクトの持つデータを
一括してこの行列(M)で変換する事ができます。

\begin{verbatim}
void transform (VertexArray* in, float* out, bool w) const;
\end{verbatim}


第1引数inにはベクトルデータを、第2引数outには結果を書き込むfloatの領域を、
第3引数wにはVertexArrayが4番目のコンポーネントを持たないときの補間方法をboolで指定します。
w=trueのとき1で補完され,w=falseの時0で補完されます。
2,3番目のコンポーネントが存在しないときは0で補完されます。



